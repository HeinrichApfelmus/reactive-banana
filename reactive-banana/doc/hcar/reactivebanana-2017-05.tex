% reactivebanana-Hr.tex
\begin{hcarentry}[updated]{reactive-banana}
\members{Heinrich Apfelmus, Oliver Charles}
\report{Heinrich Apfelmus}%05/17
\release{1.1.0.1}
\status{stable}
\makeheader

%**<img width=200 src="./banana.jpg">
%*ignore
\begin{center}
\includegraphics[width=0.2\columnwidth]{html/banana.jpg}
\end{center}
%*endignore

Reactive-banana is a library for functional reactive programming (FRP).
%
FRP offers an elegant and concise way to express interactive programs such as
graphical user interfaces, animations, computer music or robot controllers.
It promises to avoid the spaghetti code that is all too common in traditional
approaches to GUI programming.

The goal of the library is to provide a solid foundation.

\begin{compactitem}
\item Programmers interested in implementing FRP will have a \emph{reference}
  for a \emph{simple semantics} with a working implementation. The library
  stays close to the semantics pioneered by Conal Elliott.
\item The library features an \emph{efficient implementation}. No more spooky
  time leaks, predicting space \& time usage should be straightforward.
\end{compactitem}

The library is meant to be used in conjunction with existing libraries that
are specific to your problem domain. For instance, you can hook it into any
event-based GUI framework, like wxHaskell or Gtk2Hs. Several helper packages
like reactive-banana-wx provide a small amount of glue code that can make life
easier.

\subsubsection*{Status}

Having reached the milestone of version \verb!1.*!, I consider the library API
to be stable and feature complete. Compared to the previous report, thanks to the efforts of our new co-maintainer Oliver Charles, the library is now available on Stackage~\cref{Stackage}.

\subsubsection*{Future development}

There still remains some work to be done to improve the constant factor performance of the library.

Also, the library does not yet compile well to JavaScript with GHCJS, as there are some issues with garbage collection.

\FurtherReading
\begin{compactitem}
\item Project homepage: \url{http://wiki.haskell.org/Reactive-banana}
\item Example code: \url{http://wiki.haskell.org/Reactive-banana/Examples}
\end{compactitem}
\end{hcarentry}
